% Created by tikzDevice version 0.12.6 on 2025-07-29 13:52:01
% !TEX encoding = UTF-8 Unicode
\documentclass[10pt]{article}

\nonstopmode

\usepackage{tikz}

\usepackage[active,tightpage,psfixbb]{preview}

\PreviewEnvironment{pgfpicture}

\setlength\PreviewBorder{0pt}
\begin{document}

\begin{tikzpicture}[x=1pt,y=1pt]
\definecolor{fillColor}{RGB}{255,255,255}
\path[use as bounding box,fill=fillColor,fill opacity=0.00] (0,0) rectangle (505.89,202.36);
\begin{scope}
\path[clip] (  0.00,  0.00) rectangle (252.94,202.36);
\definecolor{drawColor}{RGB}{255,255,255}
\definecolor{fillColor}{RGB}{255,255,255}

\path[draw=drawColor,line width= 0.6pt,line join=round,line cap=round,fill=fillColor] (  0.00,  0.00) rectangle (252.94,202.36);
\end{scope}
\begin{scope}
\path[clip] ( 36.11, 30.69) rectangle (247.44,196.86);
\definecolor{fillColor}{RGB}{255,255,255}

\path[fill=fillColor] ( 36.11, 30.69) rectangle (247.44,196.86);
\definecolor{drawColor}{gray}{0.92}

\path[draw=drawColor,line width= 0.3pt,line join=round] ( 36.11, 58.61) --
	(247.44, 58.61);

\path[draw=drawColor,line width= 0.3pt,line join=round] ( 36.11, 99.35) --
	(247.44, 99.35);

\path[draw=drawColor,line width= 0.3pt,line join=round] ( 36.11,140.09) --
	(247.44,140.09);

\path[draw=drawColor,line width= 0.3pt,line join=round] ( 36.11,180.83) --
	(247.44,180.83);

\path[draw=drawColor,line width= 0.3pt,line join=round] ( 45.30, 30.69) --
	( 45.30,196.86);

\path[draw=drawColor,line width= 0.3pt,line join=round] ( 87.25, 30.69) --
	( 87.25,196.86);

\path[draw=drawColor,line width= 0.3pt,line join=round] (129.19, 30.69) --
	(129.19,196.86);

\path[draw=drawColor,line width= 0.3pt,line join=round] (171.14, 30.69) --
	(171.14,196.86);

\path[draw=drawColor,line width= 0.3pt,line join=round] (213.09, 30.69) --
	(213.09,196.86);

\path[draw=drawColor,line width= 0.6pt,line join=round] ( 36.11, 38.24) --
	(247.44, 38.24);

\path[draw=drawColor,line width= 0.6pt,line join=round] ( 36.11, 78.98) --
	(247.44, 78.98);

\path[draw=drawColor,line width= 0.6pt,line join=round] ( 36.11,119.72) --
	(247.44,119.72);

\path[draw=drawColor,line width= 0.6pt,line join=round] ( 36.11,160.46) --
	(247.44,160.46);

\path[draw=drawColor,line width= 0.6pt,line join=round] ( 66.27, 30.69) --
	( 66.27,196.86);

\path[draw=drawColor,line width= 0.6pt,line join=round] (108.22, 30.69) --
	(108.22,196.86);

\path[draw=drawColor,line width= 0.6pt,line join=round] (150.17, 30.69) --
	(150.17,196.86);

\path[draw=drawColor,line width= 0.6pt,line join=round] (192.12, 30.69) --
	(192.12,196.86);

\path[draw=drawColor,line width= 0.6pt,line join=round] (234.06, 30.69) --
	(234.06,196.86);
\definecolor{fillColor}{gray}{0.35}

\path[fill=fillColor] ( 45.72, 38.24) rectangle ( 53.27, 45.74);

\path[fill=fillColor] ( 54.11, 38.24) rectangle ( 61.66, 50.79);

\path[fill=fillColor] ( 62.50, 38.24) rectangle ( 70.05, 55.51);

\path[fill=fillColor] ( 70.89, 38.24) rectangle ( 78.44, 61.87);

\path[fill=fillColor] ( 79.28, 38.24) rectangle ( 86.83, 70.02);

\path[fill=fillColor] ( 87.67, 38.24) rectangle ( 95.22, 96.58);

\path[fill=fillColor] ( 96.05, 38.24) rectangle (103.61,109.94);

\path[fill=fillColor] (104.44, 38.24) rectangle (112.00,107.17);

\path[fill=fillColor] (112.83, 38.24) rectangle (120.38,144.00);

\path[fill=fillColor] (121.22, 38.24) rectangle (128.77,140.74);

\path[fill=fillColor] (129.61, 38.24) rectangle (137.16,168.61);

\path[fill=fillColor] (138.00, 38.24) rectangle (145.55,170.40);

\path[fill=fillColor] (146.39, 38.24) rectangle (153.94,184.58);

\path[fill=fillColor] (154.78, 38.24) rectangle (162.33,170.24);

\path[fill=fillColor] (163.17, 38.24) rectangle (170.72,189.30);

\path[fill=fillColor] (171.56, 38.24) rectangle (179.11,181.97);

\path[fill=fillColor] (179.95, 38.24) rectangle (187.50,179.04);

\path[fill=fillColor] (188.34, 38.24) rectangle (195.89,175.45);

\path[fill=fillColor] (196.73, 38.24) rectangle (204.28,177.57);

\path[fill=fillColor] (205.12, 38.24) rectangle (212.67,174.80);

\path[fill=fillColor] (213.51, 38.24) rectangle (221.06,169.26);

\path[fill=fillColor] (221.90, 38.24) rectangle (229.45,182.62);

\path[fill=fillColor] (230.29, 38.24) rectangle (237.84,121.19);
\definecolor{drawColor}{gray}{0.20}

\path[draw=drawColor,line width= 0.6pt,line join=round,line cap=round] ( 36.11, 30.69) rectangle (247.44,196.86);
\end{scope}
\begin{scope}
\path[clip] (  0.00,  0.00) rectangle (505.89,202.36);
\definecolor{drawColor}{gray}{0.30}

\node[text=drawColor,anchor=base east,inner sep=0pt, outer sep=0pt, scale=  0.88] at ( 31.16, 35.21) {0};

\node[text=drawColor,anchor=base east,inner sep=0pt, outer sep=0pt, scale=  0.88] at ( 31.16, 75.95) {250};

\node[text=drawColor,anchor=base east,inner sep=0pt, outer sep=0pt, scale=  0.88] at ( 31.16,116.69) {500};

\node[text=drawColor,anchor=base east,inner sep=0pt, outer sep=0pt, scale=  0.88] at ( 31.16,157.43) {750};
\end{scope}
\begin{scope}
\path[clip] (  0.00,  0.00) rectangle (505.89,202.36);
\definecolor{drawColor}{gray}{0.20}

\path[draw=drawColor,line width= 0.6pt,line join=round] ( 33.36, 38.24) --
	( 36.11, 38.24);

\path[draw=drawColor,line width= 0.6pt,line join=round] ( 33.36, 78.98) --
	( 36.11, 78.98);

\path[draw=drawColor,line width= 0.6pt,line join=round] ( 33.36,119.72) --
	( 36.11,119.72);

\path[draw=drawColor,line width= 0.6pt,line join=round] ( 33.36,160.46) --
	( 36.11,160.46);
\end{scope}
\begin{scope}
\path[clip] (  0.00,  0.00) rectangle (505.89,202.36);
\definecolor{drawColor}{gray}{0.20}

\path[draw=drawColor,line width= 0.6pt,line join=round] ( 66.27, 27.94) --
	( 66.27, 30.69);

\path[draw=drawColor,line width= 0.6pt,line join=round] (108.22, 27.94) --
	(108.22, 30.69);

\path[draw=drawColor,line width= 0.6pt,line join=round] (150.17, 27.94) --
	(150.17, 30.69);

\path[draw=drawColor,line width= 0.6pt,line join=round] (192.12, 27.94) --
	(192.12, 30.69);

\path[draw=drawColor,line width= 0.6pt,line join=round] (234.06, 27.94) --
	(234.06, 30.69);
\end{scope}
\begin{scope}
\path[clip] (  0.00,  0.00) rectangle (505.89,202.36);
\definecolor{drawColor}{gray}{0.30}

\node[text=drawColor,anchor=base,inner sep=0pt, outer sep=0pt, scale=  0.88] at ( 66.27, 19.68) {2005};

\node[text=drawColor,anchor=base,inner sep=0pt, outer sep=0pt, scale=  0.88] at (108.22, 19.68) {2010};

\node[text=drawColor,anchor=base,inner sep=0pt, outer sep=0pt, scale=  0.88] at (150.17, 19.68) {2015};

\node[text=drawColor,anchor=base,inner sep=0pt, outer sep=0pt, scale=  0.88] at (192.12, 19.68) {2020};

\node[text=drawColor,anchor=base,inner sep=0pt, outer sep=0pt, scale=  0.88] at (234.06, 19.68) {2025};
\end{scope}
\begin{scope}
\path[clip] (  0.00,  0.00) rectangle (505.89,202.36);
\definecolor{drawColor}{RGB}{0,0,0}

\node[text=drawColor,anchor=base,inner sep=0pt, outer sep=0pt, scale=  1.10] at (141.78,  7.64) {Year};
\end{scope}
\begin{scope}
\path[clip] (  0.00,  0.00) rectangle (505.89,202.36);
\definecolor{drawColor}{RGB}{0,0,0}

\node[text=drawColor,rotate= 90.00,anchor=base,inner sep=0pt, outer sep=0pt, scale=  1.10] at ( 13.08,113.77) {Citations/Year};
\end{scope}
\begin{scope}
\path[clip] (252.94,  0.00) rectangle (505.89,202.36);
\definecolor{drawColor}{RGB}{255,255,255}
\definecolor{fillColor}{RGB}{255,255,255}

\path[draw=drawColor,line width= 0.6pt,line join=round,line cap=round,fill=fillColor] (252.94,  0.00) rectangle (505.89,202.36);
\end{scope}
\begin{scope}
\path[clip] (297.85, 30.69) rectangle (500.39,196.86);
\definecolor{fillColor}{RGB}{255,255,255}

\path[fill=fillColor] (297.85, 30.69) rectangle (500.39,196.86);
\definecolor{drawColor}{gray}{0.92}

\path[draw=drawColor,line width= 0.3pt,line join=round] (297.85, 65.11) --
	(500.39, 65.11);

\path[draw=drawColor,line width= 0.3pt,line join=round] (297.85,118.86) --
	(500.39,118.86);

\path[draw=drawColor,line width= 0.3pt,line join=round] (297.85,172.60) --
	(500.39,172.60);

\path[draw=drawColor,line width= 0.3pt,line join=round] (306.66, 30.69) --
	(306.66,196.86);

\path[draw=drawColor,line width= 0.3pt,line join=round] (346.86, 30.69) --
	(346.86,196.86);

\path[draw=drawColor,line width= 0.3pt,line join=round] (387.06, 30.69) --
	(387.06,196.86);

\path[draw=drawColor,line width= 0.3pt,line join=round] (427.26, 30.69) --
	(427.26,196.86);

\path[draw=drawColor,line width= 0.3pt,line join=round] (467.46, 30.69) --
	(467.46,196.86);

\path[draw=drawColor,line width= 0.6pt,line join=round] (297.85, 38.24) --
	(500.39, 38.24);

\path[draw=drawColor,line width= 0.6pt,line join=round] (297.85, 91.98) --
	(500.39, 91.98);

\path[draw=drawColor,line width= 0.6pt,line join=round] (297.85,145.73) --
	(500.39,145.73);

\path[draw=drawColor,line width= 0.6pt,line join=round] (326.76, 30.69) --
	(326.76,196.86);

\path[draw=drawColor,line width= 0.6pt,line join=round] (366.96, 30.69) --
	(366.96,196.86);

\path[draw=drawColor,line width= 0.6pt,line join=round] (407.16, 30.69) --
	(407.16,196.86);

\path[draw=drawColor,line width= 0.6pt,line join=round] (447.36, 30.69) --
	(447.36,196.86);

\path[draw=drawColor,line width= 0.6pt,line join=round] (487.57, 30.69) --
	(487.57,196.86);
\definecolor{fillColor}{gray}{0.35}

\path[fill=fillColor] (307.06, 38.24) rectangle (314.30, 41.53);

\path[fill=fillColor] (315.10, 38.24) rectangle (322.34, 42.36);

\path[fill=fillColor] (323.14, 38.24) rectangle (330.38, 43.50);

\path[fill=fillColor] (331.18, 38.24) rectangle (338.42, 45.05);

\path[fill=fillColor] (339.22, 38.24) rectangle (346.46, 47.15);

\path[fill=fillColor] (347.26, 38.24) rectangle (354.50, 51.00);

\path[fill=fillColor] (355.30, 38.24) rectangle (362.54, 55.73);

\path[fill=fillColor] (363.34, 38.24) rectangle (370.58, 60.27);

\path[fill=fillColor] (371.38, 38.24) rectangle (378.62, 67.25);

\path[fill=fillColor] (379.42, 38.24) rectangle (386.66, 74.01);

\path[fill=fillColor] (387.46, 38.24) rectangle (394.70, 82.61);

\path[fill=fillColor] (395.50, 38.24) rectangle (402.74, 91.33);

\path[fill=fillColor] (403.54, 38.24) rectangle (410.78,100.98);

\path[fill=fillColor] (411.58, 38.24) rectangle (418.82,109.69);

\path[fill=fillColor] (419.62, 38.24) rectangle (426.86,119.65);

\path[fill=fillColor] (427.67, 38.24) rectangle (434.90,129.13);

\path[fill=fillColor] (435.71, 38.24) rectangle (442.94,138.42);

\path[fill=fillColor] (443.75, 38.24) rectangle (450.98,147.47);

\path[fill=fillColor] (451.79, 38.24) rectangle (459.02,156.66);

\path[fill=fillColor] (459.83, 38.24) rectangle (467.06,165.67);

\path[fill=fillColor] (467.87, 38.24) rectangle (475.10,174.31);

\path[fill=fillColor] (475.91, 38.24) rectangle (483.14,183.83);

\path[fill=fillColor] (483.95, 38.24) rectangle (491.18,189.30);
\definecolor{drawColor}{gray}{0.20}

\path[draw=drawColor,line width= 0.6pt,line join=round,line cap=round] (297.85, 30.69) rectangle (500.39,196.86);
\end{scope}
\begin{scope}
\path[clip] (  0.00,  0.00) rectangle (505.89,202.36);
\definecolor{drawColor}{gray}{0.30}

\node[text=drawColor,anchor=base east,inner sep=0pt, outer sep=0pt, scale=  0.88] at (292.90, 35.21) {0};

\node[text=drawColor,anchor=base east,inner sep=0pt, outer sep=0pt, scale=  0.88] at (292.90, 88.95) {5000};

\node[text=drawColor,anchor=base east,inner sep=0pt, outer sep=0pt, scale=  0.88] at (292.90,142.70) {10000};
\end{scope}
\begin{scope}
\path[clip] (  0.00,  0.00) rectangle (505.89,202.36);
\definecolor{drawColor}{gray}{0.20}

\path[draw=drawColor,line width= 0.6pt,line join=round] (295.10, 38.24) --
	(297.85, 38.24);

\path[draw=drawColor,line width= 0.6pt,line join=round] (295.10, 91.98) --
	(297.85, 91.98);

\path[draw=drawColor,line width= 0.6pt,line join=round] (295.10,145.73) --
	(297.85,145.73);
\end{scope}
\begin{scope}
\path[clip] (  0.00,  0.00) rectangle (505.89,202.36);
\definecolor{drawColor}{gray}{0.20}

\path[draw=drawColor,line width= 0.6pt,line join=round] (326.76, 27.94) --
	(326.76, 30.69);

\path[draw=drawColor,line width= 0.6pt,line join=round] (366.96, 27.94) --
	(366.96, 30.69);

\path[draw=drawColor,line width= 0.6pt,line join=round] (407.16, 27.94) --
	(407.16, 30.69);

\path[draw=drawColor,line width= 0.6pt,line join=round] (447.36, 27.94) --
	(447.36, 30.69);

\path[draw=drawColor,line width= 0.6pt,line join=round] (487.57, 27.94) --
	(487.57, 30.69);
\end{scope}
\begin{scope}
\path[clip] (  0.00,  0.00) rectangle (505.89,202.36);
\definecolor{drawColor}{gray}{0.30}

\node[text=drawColor,anchor=base,inner sep=0pt, outer sep=0pt, scale=  0.88] at (326.76, 19.68) {2005};

\node[text=drawColor,anchor=base,inner sep=0pt, outer sep=0pt, scale=  0.88] at (366.96, 19.68) {2010};

\node[text=drawColor,anchor=base,inner sep=0pt, outer sep=0pt, scale=  0.88] at (407.16, 19.68) {2015};

\node[text=drawColor,anchor=base,inner sep=0pt, outer sep=0pt, scale=  0.88] at (447.36, 19.68) {2020};

\node[text=drawColor,anchor=base,inner sep=0pt, outer sep=0pt, scale=  0.88] at (487.57, 19.68) {2025};
\end{scope}
\begin{scope}
\path[clip] (  0.00,  0.00) rectangle (505.89,202.36);
\definecolor{drawColor}{RGB}{0,0,0}

\node[text=drawColor,anchor=base,inner sep=0pt, outer sep=0pt, scale=  1.10] at (399.12,  7.64) {Year};
\end{scope}
\begin{scope}
\path[clip] (  0.00,  0.00) rectangle (505.89,202.36);
\definecolor{drawColor}{RGB}{0,0,0}

\node[text=drawColor,rotate= 90.00,anchor=base,inner sep=0pt, outer sep=0pt, scale=  1.10] at (266.02,113.77) {Cumulative Citations};
\end{scope}
\end{tikzpicture}

\end{document}
